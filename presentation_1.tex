%%%%%%%%%%%%%%%%%%%%%%%%%%%%%%%%%%%%%%%%%
% Beamer Presentation
% LaTeX Template
% Version 1.0 (10/11/12)
%
% This template has been downloaded from:
% http://www.LaTeXTemplates.com
%
% License:
% CC BY-NC-SA 3.0 \ttp://creativecommons.org/licenses/by-nc-sa/3.0/)
%
%%%%%%%%%%%%%%%%%%%%%%%%%%%%%%%%%%%%%%%%%

%----------------------------------------------------------------------------------------
%	PACKAGES AND THEMES
%----------------------------------------------------------------------------------------

\documentclass[12pt]{beamer}
\usepackage[absolute,overlay]{textpos}

% small logo down the page
\logo{\makebox[1\paperwidth]{\includegraphics[width=.5cm,keepaspectratio]{images/sapienza-logo.png}}}

%\fontsize{12pt}{10}\selectfont



\usepackage{tikz}
%\setbeamertemplate{background canvas}{

%\begin{tikzpicture}
%\node[opacity=.1]{
%\includegraphics[width=2in , height=3in, keepaspectratio]%{images/sapienza-logo.png}};
%\end{tikzpicture}
%}
 % only for the image: http://ctan.org/pkg/mwe
%     \setbeamertemplate{background}{\includegraphics[width=\paperwidth]{images/sapienza-logo.png}}
%{\usebackgroundtemplate{
%       \begin{picture}
%            \includegraphics[width=\paperwidth]{images/sapienza-logo.png}
%       \end{picture}
%}%


% watermark logo , you can choose width and position in a pretty accurate way
%"anchor" and current page."x" play the game : north , south , etc.
\setbeamertemplate{background}{\tikz[overlay,remember picture, anchor=south]\node[opacity=.15]at ([yshift=1.5cm]current page.south){\includegraphics[width=3cm]{images/sapienza-logo.png}};}


\definecolor{color1}{HTML}{001000} % Color of the article title and sections
\definecolor{color2}{HTML}{FF7a01} % Color of the boxes behind the abstract and headings
\definecolor{color3}{HTML}{202020} % Background color

\definecolor{baseColor}{HTML}{FF7a01} % Color of the boxes behind the abstract and headings
\mode<presentation> {

% The Beamer class comes with a number of default slide themes
% which change the colors and layouts of slides. Below this is a list
% of all the themes, uncomment each in turn to see what they look like.


%   \usetheme{Frankfurt}
%  \usetheme{Ilmenau}
%  \usetheme{JuanLesPins}
%  \usetheme{Luebeck}
%   \usetheme{PaloAlto}

\usetheme{Rochester}


% As well as themes, the Beamer class has a number of color themes
% for any slide theme. Uncomment each of these in turn to see how it
% changes the colors of your current slide theme.


% \usecolortheme{crane}
%\usecolortheme{orchid}
%   \usecolortheme{rose}
%  \usecolortheme{whale}


%\setbeamertemplate{footline} % To remove the footer line in all slides uncomment this line
\setbeamertemplate{footline}[page number] % To replace the footer line in all slides with a simple slide count uncomment this line

\setbeamertemplate{navigation symbols}{} % To remove the navigation symbols from the bottom of all slides uncomment this line

 \setbeamercolor*{palette primary}{use=structure,fg=color2,bg=color2!110 }
 \setbeamercolor*{palette secondary}{use=structure,fg=color3,bg=color3}
 \setbeamercolor{title}{fg=white,bg=baseColor!50!black}


}


\usepackage{graphicx} % Allows including images
\usepackage{booktabs} % Allows the use of \toprule, \midrule and \bottomrule in tables
\usepackage{scrextend}
\changefontsizes{11pt}
%----------------------------------------------------------------------------------------
%	TITLE PAGE
%----------------------------------------------------------------------------------------
\title[Short title]{Automated model-based android GUI testing using multi-level GUI Comparison Criteria} % The short title appears at the bottom of every slide, the full title is only on the title page

\author{happyBiker, haxxorrman} % Your name
\institute[Sapienza] % Your institution as it will appear on the bottom of every slide, may be shorthand to save space
{
Flatulenza, University of Prone \\ % Your institution for the title page
\medskip
\textit{homemadebike.9999999@stud.unipronax.fritt,} % Your email address
\textit{lexusGTsportage.99999999@stud.unipronax.fritt}

}
\date{\today} % Date, can be changed to a custom date

%%%%%%%%%%%%%%%%%%%%%%%%%%%%%%%%%%%%%%%%%%%%%%%%%
%%%   New style here

\mode<presentation>{\usetheme{AnnArbor}}
\usecolortheme{whale}

\setbeamercolor{frametitle}{parent=subsection in head/foot}
\setbeamercolor{frametitle right}{parent=section in head/foot}

\makeatletter
\pgfdeclarehorizontalshading[frametitle.bg,frametitle right.bg]{beamer@frametitleshade}{\paperheight}{%
    color(0pt)=(frametitle.bg);
    color(\paperwidth)=(frametitle right.bg)}

\AtBeginDocument{
    \pgfdeclareverticalshading{beamer@topshade}{\paperwidth}{%
        color(0pt)=(bg);
        color(4pt)=(black!50!bg)}
}

\addtobeamertemplate{headline}
{}
{%
    \vskip-0.2pt
    \pgfuseshading{beamer@topshade}
    \vskip-2pt
}


\setbeamertemplate{frametitle}
{%
    \nointerlineskip%
    \vskip-2pt%
    \hbox{\leavevmode
        \advance\beamer@leftmargin by -12bp%
        \advance\beamer@rightmargin by -12bp%
        \beamer@tempdim=\textwidth%
        \advance\beamer@tempdim by \beamer@leftmargin%
        \advance\beamer@tempdim by \beamer@rightmargin%
        \hskip-\Gm@lmargin\hbox{%
            \setbox\beamer@tempbox=\hbox{\begin{minipage}[b]{\paperwidth}%
                    \vbox{}\vskip-.75ex%
                    \leftskip0.3cm%
                    \rightskip0.3cm plus1fil\leavevmode
                    \insertframetitle%
                    \ifx\insertframesubtitle\@empty%
                    \strut\par%
                    \else
                    \par{\usebeamerfont*{framesubtitle}{\usebeamercolor[fg]{framesubtitle}\insertframesubtitle}\strut\par}%
                    \fi%
                    \nointerlineskip
                    \vbox{}%
                \end{minipage}}%
                \beamer@tempdim=\ht\beamer@tempbox%
                \advance\beamer@tempdim by 2pt%
                \begin{pgfpicture}{0pt}{0pt}{\paperwidth}{\beamer@tempdim}
                    \usebeamercolor{frametitle right}
                    \pgfpathrectangle{\pgfpointorigin}{\pgfpoint{\paperwidth}{\beamer@tempdim}}
                    \pgfusepath{clip}
                    \pgftext[left,base]{\pgfuseshading{beamer@frametitleshade}}
                \end{pgfpicture}
                \hskip-\paperwidth%
                \box\beamer@tempbox%
            }%
            \hskip-\Gm@rmargin%
        }%
        \nointerlineskip
        \vskip-0.2pt
        \hbox to\textwidth{\hskip-\Gm@lmargin\pgfuseshading{beamer@topshade}\hskip-\Gm@rmargin}
        \vskip-2pt
    }
\makeatother

\setbeamercolor{section in toc}{fg=red}
%%%

\setbeamercolor{structure}{fg=baseColor!80!black}
\setbeamercolor*{block title example}{fg=blue!50,bg= blue!10}
\setbeamercolor*{block body example}{fg= red,bg= blue!5}
\usefonttheme{structuresmallcapsserif}

\setbeamertemplate{footline}[page number]{} 

\setbeamercolor{headline}{fg=blue!90!black,bg=baseColor!90!black}
\setbeamercolor{palette primary}{fg=white,bg=baseColor!90!black}
\setbeamercolor{palette secondary}{fg=white,bg=baseColor!90!black}
\setbeamercolor{palette tertiary}{fg=white,bg=black}
\setbeamercolor{frametitle}{fg=white,bg=baseColor!50!black}


\colorlet{titleleft}{baseColor}
\colorlet{titleright}{black}

\setbeamercolor*{frametitle}{fg=white}

\makeatletter
\pgfdeclarehorizontalshading[titleleft,titleright]{beamer@frametitleshade}{\paperheight}{%
    color(0pt)=(titleleft);
    color(\paperwidth)=(titleright)}
\makeatother

%%%
%%%  End new style
%%%%%%%%%%%%%%%%%%%%%%%%%%%%%%%%%%%%%%%%%%%%%%%%




\begin{document}

\begin{frame}
\titlepage % Print the title page as the first slide
\end{frame}

\begin{frame}
\frametitle{Overview} % Table of contents slide, comment this block out to remove it
\tableofcontents % Throughout your presentation, if you choose to use \section{} and \subsection{} commands, these will automatically be printed on this slide as an overview of your presentation
\end{frame}

%----------------------------------------------------------------------------------------
%	PRESENTATION SLIDES
%----------------------------------------------------------------------------------------

%------------------------------------------------
\section{Introduzione} % Sections can be created in order to organize your presentation into discrete blocks, all sections and subsections are automatically printed in the table of contents as an overview of the talk
%------------------------------------------------

\begin{frame}
    \frametitle{Introduzione}
    In questa presentazione viene analizzato e riassunto il contentuto dell'articolo \textbf{``Automated Model-Based Android GUI Testing
    using Multi-level GUI Comparison Criteria''} scritto da Young-Min Baek e Doo-Hwan Bae.
\\~\\
Presentazione preparata da 

\end{frame}

\begin{frame}
    \frametitle{GUI testing automatico(1)}
    \begin{block}{Perch\'e dedicare particolare attenzione alle applicazioni Android?}
        Attualmente le applicazioni android comprendono quasi il 90\% del mercato di software mobile.

    \end{block}
    \begin{block}{Cos'\`e il GUI testing?}
        \`E una metodologia di testing che esplora una parte dello spazio degli stati del programma testato dando gli input al programma tramite l'interfaccia grafica.
    \end{block}
    \begin{block}{Perch\'e automatizzare il testing?}
        Il motivo principale \`e il costo, l'esecuzione manuale dei test ha il costo eccessivo per molte applicazioni.
        Inoltre l'esecuzione manuale dei test richiede tanto tempo e a volte non \`e in grado di individuare tutti gli errori.
    \end{block}


\end{frame}

\begin{frame}
    \frametitle{GUI testing automatico(2)}
Attualmente sul mercato si trovano dei tool per il GUI testing automatico che generano gli input in modo alleatorio, tentando di generare una sequenza di input che provoca il malfunzionamento del software.
Nel testo viene citato un tool chiamato \emph{Android Monkey} che permette questo tipo di test.
\\~\\
Dato che si tratta essenzialmente di una ricerca locale nello spazio degli stati, questi metodi sono soggetti alle problematiche degli algoritmi di ricerca locale. Per esempio possono valutare lo stesso stato pi\`u volte.
\\~\\
L'approccio proposto dagli autori \`e di costruire un grafo degli stati della GUI, utilizzando dei criteri di equivalenza (GUICC) per distinguere gli stati.
\end{frame}



\subsection{Subsection Example} % A subsection can be created just before a set of slides with a common theme to further break down your presentation into chunks

\begin{frame}
\frametitle{Paragraphs of Text}
Sed iaculis dapibus gravida. Morbi sed tortor erat, nec interdum arcu. Sed id lorem lectus. Quisque viverra augue id sem ornare non aliquam nibh tristique. Aenean in ligula nisl. Nulla sed tellus ipsum. Donec vestibulum ligula non lorem vulputate fermentum accumsan neque mollis.\\~\\

Sed diam enim, sagittis nec condimentum sit amet, ullamcorper sit amet libero. Aliquam vel dui orci, a porta odio. Nullam id suscipit ipsum. Aenean lobortis commodo sem, ut commodo leo gravida vitae. Pellentesque vehicula ante iaculis arcu pretium rutrum eget sit amet purus. Integer ornare nulla quis neque ultrices lobortis. Vestibulum ultrices tincidunt libero, quis commodo erat ullamcorper id.
\end{frame}

%------------------------------------------------

\begin{frame}
\frametitle{Bullet Points}
\begin{itemize}
\item Lorem ipsum dolor sit amet, consectetur adipiscing elit
\item Aliquam blandit faucibus nisi, sit amet dapibus enim tempus eu
\item Nulla commodo, erat quis gravida posuere, elit lacus lobortis est, quis porttitor odio mauris at libero
\item Nam cursus est eget velit posuere pellentesque
\item Vestibulum faucibus velit a augue condimentum quis convallis nulla gravida
\end{itemize}
\end{frame}

%------------------------------------------------

\begin{frame}
\frametitle{Blocks of Highlighted Text}
\begin{block}{Block 1}
Lorem ipsum dolor sit amet, consectetur adipiscing elit. Integer lectus nisl, ultricies in feugiat rutrum, porttitor sit amet augue. Aliquam ut tortor mauris. Sed volutpat ante purus, quis accumsan dolor.
\end{block}

\begin{block}{Block 2}
Pellentesque sed tellus purus. Class aptent taciti sociosqu ad litora torquent per conubia nostra, per inceptos himenaeos. Vestibulum quis magna at risus dictum tempor eu vitae velit.
\end{block}

\begin{block}{Block 3}
Suspendisse tincidunt sagittis gravida. Curabitur condimentum, enim sed venenatis rutrum, ipsum neque consectetur orci, sed blandit justo nisi ac lacus.
\end{block}
\end{frame}

%------------------------------------------------

\begin{frame}
\frametitle{Multiple Columns}
\begin{columns}[c] % The "c" option specifies centered vertical alignment while the "t" option is used for top vertical alignment

\column{.45\textwidth} % Left column and width
\textbf{Heading}
\begin{enumerate}
\item Statement
\item Explanation
\item Example
\end{enumerate}

\column{.5\textwidth} % Right column and width
Lorem ipsum dolor sit amet, consectetur adipiscing elit. Integer lectus nisl, ultricies in feugiat rutrum, porttitor sit amet augue. Aliquam ut tortor mauris. Sed volutpat ante purus, quis accumsan dolor.

\end{columns}
\end{frame}

%------------------------------------------------
\section{Second Section}
%------------------------------------------------

\begin{frame}
\frametitle{Table}
\begin{table}
\begin{tabular}{l l l}
\toprule
\textbf{Treatments} & \textbf{Response 1} & \textbf{Response 2}\\
\midrule
Treatment 1 & 0.0003262 & 0.562 \\
Treatment 2 & 0.0015681 & 0.910 \\
Treatment 3 & 0.0009271 & 0.296 \\
\bottomrule
\end{tabular}
\caption{Table caption}
\end{table}
\end{frame}

%------------------------------------------------

\begin{frame}
\frametitle{Theorem}
\begin{theorem}[Mass--energy equivalence]
$E = mc^2$
\end{theorem}
\end{frame}

%------------------------------------------------

\begin{frame}[fragile] % Need to use the fragile option when verbatim is used in the slide
\frametitle{Verbatim}
\begin{example}[Theorem Slide Code]
\begin{verbatim}
\begin{frame}
\frametitle{Theorem}
\begin{theorem}[Mass--energy equivalence]
$E = mc^2$
\end{theorem}
\end{frame}\end{verbatim}
\end{example}
\end{frame}

%------------------------------------------------

\begin{frame}
\frametitle{Figure}
Uncomment the code on this slide to include your own image from the same directory as the template .TeX file.
%\begin{figure}
%\includegraphics[width=0.8\linewidth]{test}
%\end{figure}
\end{frame}

%------------------------------------------------

\begin{frame}[fragile] % Need to use the fragile option when verbatim is used in the slide
\frametitle{Citation}
An example of the \verb|\cite| command to cite within the presentation:\\~

This statement requires citation \cite{p1}.
\end{frame}

%------------------------------------------------

\begin{frame}
\frametitle{References}
\footnotesize{
\begin{thebibliography}{99} % Beamer does not support BibTeX so references must be inserted manually as below
\bibitem[Smith, 2012]{p1} John Smith (2012)
\newblock Title of the publication
\newblock \emph{Journal Name} 12(3), 45 -- 678.
\end{thebibliography}
}
\end{frame}

%------------------------------------------------

\begin{frame}
\Huge{\centerline{The End}}
\end{frame}

%----------------------------------------------------------------------------------------

\end{document} 
